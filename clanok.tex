% Metódy inžinierskej práce
% Dávid Schmidt

\documentclass[10pt,oneside,slovak,a4paper]{article}

\usepackage[slovak]{babel}
%\usepackage[T1]{fontenc}
\usepackage[IL2]{fontenc} 
\usepackage[utf8]{inputenc}
\usepackage{graphicx}
\usepackage{url} 
\usepackage{hyperref} 

\usepackage{cite}
%\usepackage{times}

\pagestyle{headings}

\title{Duálne vzdelávanie na Slovensku\thanks{Semestrálny projekt v predmete Metódy inžinierskej práce, ak. rok 2020/21, vedenie: Ing. Zuzana Špitálová}} 

\author{Dávid Schmidt\\[2pt]
	{\small Slovenská technická univerzita v Bratislave}\\
	{\small Fakulta informatiky a informačných technológií}\\
	{\small \texttt{xschmidtd@stuba.sk}}
	}

\date{\small 8. november 2020} 



\begin{document}

\maketitle

\begin{abstract}
Duálne vzdelávanie na Slovensku si v posledných rokoch získalo popularitu medzi študentami aj ich budúcimi potenciálnymi zamestnávateľmi. Za svoju obľúbenosť medzi študentami vďačí prísľubom istého zamestnania po skončení štúdia. Na druhej strane sa nachádza trh práce a teda firmy, ktoré majú istotu vzdelaného študenta s požadovanými vedomosťami v radoch vlastných zamestnancoch. Článok je zameraný na samotný proces duálneho vzdelávania, jeho výhody a nedostatky. Taktiež sa venuje otázke, či duálne vzdelávanie dokáže kvalitatívne konkurovať vysokým školám a teda či jeho absolventi po nástupe na trh práce sa dokážu rovnať ich vysokoškolským konkurentom v oblasti vedomostí ale aj platového ohodnotenia.

\end{abstract}


\section{Úvod}
\quad Veľký počet nezamestnaných mladých ľudí sa objavil v mnohých európskych krajinách po roku 2009. Avšak v krajinách ako Rakúsko, Dánsko, Nemecko, Švajčiarsko sa táto vlna nezamestananosti neprejavila v takom rozsahu ako vo zvyšku Európy. 
Tieto krajiny majú spoločnú rozlišovaciu črtu oproti ostatným: značná časť vzdelávacieho systému je realizovaná systémom tréningového odborného vzdelávania a teda výučba neprebieha len tradične v školách ale aj v podnikoch. Tento spôsob vzdelávania sa zdá byť mimoriadne efektívny, čo nám dokazujú aj už spomínané štatistiky nezamestnanosti. 

Mnohé slovenské podniky podceňujú skutočnosť, že kvalitní zamestanci sú kľúčovným faktorom úspechu. Personalisti sú v podnikoch nútení vyhľadávať absolventov nielen s odbornými vedomosťami, ale takých ktorý majú jazykové zručnosti a soft skills. Zamestnávatelia sa neustále stretávajú s absolventmi bohatými na teoretické vedomosti, no po nástupe do práce nevedia tieto poznatky uplatiť a musia prejsť dlhým a často nákladným procesom zaškoľovania. Na druhej strane skúsený zamestnanec vyžaduje neporovnateľne kratší čas zaškoľovania. Zamestnávatelia si často kompenzujú nedostatok skúseností a zručností nižším platovým ohodnotnením. Avšak problém nedostatočných skúseností sa tým nerieši a stále existujú pracovné pozície, kde je dokonca nemožné nastúpiť bez požadovanej praxe. Adekvátnym riešením tohto problému môže byť odborné vzdelávanie úzko späté s duálnym vzdelávaním.

\section{Princíp duálneho vzdelávania}
\quad Požiadavky kladené na duálne vzdelávanie sú jasné. Primárnym cieľom je preniesť teoretické vedomosti do praxi a vedieť ich aplikovať na riešenie konkrétnych problémov, s ktorými sa absolventi stretnú v zamestnaní a budú ich musieť riešiť čo najefektívnejšie a najoptimálnejšie. Výsledkom takéhoto vzdelávania je odborné vzdelanie, ktoré sa stáva súčasťou odbornej kvalifikácie jednotlivca. 

\section{Výhody duálneho vzdelávania}

\section{Výzvy duálneho vzdelávania}

\section{Záver} \label{zaver} % prípadne iný variant názvu

%\acknowledgement{Ak niekomu chcete poďakovať\ldots}

\nocite{*} 
\bibliography{literatura}
\bibliographystyle{alpha} 
\end{document}
